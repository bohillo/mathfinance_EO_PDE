\documentclass{article}

\usepackage{polski}
\usepackage{xcolor}
\usepackage{bbm}

\usepackage[latin2]{inputenc}

% Dane magistranta:

\author{Piotr Bochnia, Pawe� Marcinkowski}


\title{Finanse obliczeniowe - Du�y projekt \\ Wycena skomplikowanych opcji barierowych metod� PDE}



% Tu jest dobre miejsce na Twoje w�asne makra i~�rodowiska:
\newtheorem{defi}{Definicja}[section]
\newcommand{\notka}[1]{\textcolor{red}{#1}}
\newtheorem{assum}{Za�o�enie}[section]
% koniec definicji

\begin{document}
\maketitle

\clearpage
\tableofcontents
%\listoffigures
%\listoftables
\clearpage


\section{Wst�p}
Celem niniejszego projektu jest implementacja algorytm�w wyceny wybranych opcji barierowych metod� opart� na rozwi�zywaniu r�wnania Blacka-Scholesa. Na podstawie danych rynkowych oraz charakterystyk opcji napisany w Octave program wyznacza parametry r�wnania Blacka-Scholesa wraz z odpowiednimi dla danego kontraktu warunkami brzegowymi i ko�cowymi, a nast�pnie rozwi�zuje to r�wnanie metod� r�nic sko�czonych. Poza cen� opcji obliczane s� tak�e parametry greckie: \emph{delta spot}, \emph{delta forward}, \emph{gamma spot}, \emph{gamma forward}, \emph{theta}, oraz \emph{vega}.  
\section{Wyceniane instrumenty} Wycen� przeprowadzono dla wybranych
skomplikowanych opcji barierowych: opcji z pojedyncz�, dyskretnie
monitorowan� barier�, opcji z podw�jn� barier� (monitorowan� w spos�b
zar�wno dyskretny jak i ci�g�y), opcji barierowych z barierami
monitorowanymi w oknie czasowym, oraz opcji paryskich.

Poni�ej wymieniono poszczeg�lne typy kontrakt�w, dla kt�rych zaimplementowano algorytm wyceny. Niech $T$ b�dzie czasem zapadalno�ci opcji, $K$ cen� wykonania, $U$ i $L$ odpowiednio barier� g�rn� i doln�, oraz $S_t$ cen� instrumentu bazowego w chwili $t$. Ponadto niech $f$ b�dzie funkcj� wyp�aty dla opcji waniliowej tj. $f(x) = (x - K) ^{+}$ dla opcji \emph{call} i $f(x) = (K - x) ^{+}$ dla opcji \emph{put}.

\subsection{Opcje z barier� monitorowan� dyskretnie}
Niech $0 \leq T_1 < T_2 < \ldots < T_L \leq T$ b�d� punktami monitorowania bariery. 
\begin{itemize}
\item \emph{Up and out} z wyp�at� $X = f(S_T) \cdot \mathbbm{1}_{\{\forall_{ t \in \{T_1, \ldots, T_L\} } S_t < U \}}$
\item \emph{Up and in} z wyp�at� $X = f(S_T) \cdot \mathbbm{1}_{\{\exists_{ t \in \{T_1, \ldots, T_L\} } S_t \geq U \}}$
\item \emph{Down and out} z wyp�at� $X = f(S_T) \cdot \mathbbm{1}_{\{\forall_{ t \in \{T_1, \ldots, T_L\} } S_t > L \}}$
\item \emph{Down and in} z wyp�at� $X = f(S_T) \cdot \mathbbm{1}_{\{\exists_{ t \in \{T_1, \ldots, T_L\} } S_t \leq L \}}$
\item \emph{Double Knock-out} z wyp�at� $X = f(S_T) \cdot \mathbbm{1}_{\{\forall_{ t \in \{T_1, \ldots, T_L\} } L < S_t  < U \}}$. Zaimplementowano r�wnie� wycen� tego typu opcji z ci�g�ym monitorowaniem bariery.
\item \emph{Knock-in Knock-out} z wyp�at� $X = f(S_T) \cdot \left( \mathbbm{1}_{\{\exists_{ t \in \{T_1, \ldots, T_L\} } S_t \leq L \} \wedge \forall_{ t \in \{T_1, \ldots, T_L\} } S_t < U \}}\right)$. Zaimplementowano r�wnie� wycen� tego typu opcji z ci�g�ym monitorowaniem bariery.


\end{itemize}

\subsection{Opcje z barier� monitorowan� w oknie czasowym}
Niech $0 \leq \tau_1 < \tau_2 \leq T$ b�d� punktami odpowiednio pocz�tku i ko�ca okna, kt�rym monitorowana jest bariera. 
\begin{itemize}
\item \emph{Up and out} z wyp�at� $X = f(S_T) \cdot \mathbbm{1}_{\{\forall_{ t \in \left[ \tau_1, \tau_2 \right] } S_t < U \}}$
\item \emph{Up and in} z wyp�at� $X = f(S_T) \cdot \mathbbm{1}_{\{\exists_{ t \in \left[ \tau_1, \tau_2 \right] } S_t \geq U \}}$
\item \emph{Down and out} z wyp�at� $X = f(S_T) \cdot \mathbbm{1}_{\{\forall_{ t \in \left[ \tau_1, \tau_2 \right] } S_t > L \}}$
\item \emph{Down and in} z wyp�at� $X = f(S_T) \cdot \mathbbm{1}_{\{\exists_{ t \in \left[ \tau_1, \tau_2 \right] } S_t \leq L \}}$
\item \emph{Double Knock-out} z wyp�at� $X = f(S_T) \cdot \mathbbm{1}_{\{\forall_{ t \in \left[ \tau_1, \tau_2 \right] } L < S_t  < U \}}$
\item \emph{Knock-in Knock-out} z wyp�at� $X = f(S_T) \cdot \left( \mathbbm{1}_{\{\exists_{ t \in \left[ \tau_1, \tau_2 \right] } S_t \leq L \} \wedge \forall_{ t \in \left[ \tau_1, \tau_2 \right] } S_t < U \}}\right)$


\end{itemize}
\subsection{Opcje paryskie}
Jednobarierowe opcje paryskie s� kontraktami, w kt�rych w�asno�� \emph{in} lub \emph{out} jest aktywowana nie w momencie dotkni�cia bariery lecz po pewnym ustalonym z g�ry czasie przebywania ceny instrumentu bazowego nad lub pod barier�. Opcje typu paryskiego dziel� si� na dwie klasy:
\begin{itemize}
\item  \emph{Parisian} - 
\end{itemize}

\section{Dokumentacja funkcji}

Wszystkie funkcje u�ywane do wyceny opcji zwracaj� 7-elementowy wektor zawieraj�cy na kolejnych pozycjach odpowiednio cen� opcji, delt� spot, delt� forward, gamm� spot, gamm� forward, thet� oraz veg�. Funkcje przyjmuj� nast�puj�ce argumenty:

\begin{itemize}
\item F\_bid - kurs forward bid
\item F\_bid - kurs forward ask
\item barrier - wysoko�� bariery (w przypadku opcji jednobarierowych, wyra�ona w walucie kwotowania)
\item Lbarrier - wysoko�� dolnej bariery (w przypadku opcji dwubarierowych, wyra�ona w walucie kwotowania)
\item Ubarrier - wysoko�� g�rnej bariery (w przypadku opcji dwubarierowych, wyra�ona w walucie kwotowania)
\item strike - kurs wykonania opcji
\item issue\_date - data zawarcia kontraktu (np. '21-May-2014')
\item expire\_date - data zapadalno�ci kontraktu (np. '21-May-2014') 
\item PPO - liczba dni roboczych od daty zawarcia kontraktu do dnia zap�aty premi opcyjnej (\emph{Premium Payment Offset})
\item OSO - liczba dni roboczych od daty zapadalno�ci opcji do dnia rozliczenia kontraktu (\emph{Option Settlement Offset})
\item monitoring\_dates - wektor zawieraj�cy daty monitorowania barier w kolejno�ci chronologicznej
\item price\_type - typ obliczanej ceny (poprawne warto�ci: \emph{bid}, \emph{ask})
\item barrier\_type - typ bariery (w przypadku opcji jednobarierowych, poprawne warto�ci: \emph{up}, \emph{down})
\item payoff\_type - typ opcji (poprawne warto�ci: \emph{put}, \emph{call})
\item window\_start\_date - data pocz�tku okna czasowego, w kt�rym monitorowane s� bariery
\item window\_end\_date - data ko�ca okna czasowego, w kt�rym monitorowane s� bariery
\item day\_hat - czas przebywania poza barier� konieczny do aktywacji bariery (w przypadku opcji paryskich, wyra�ony w dniach)
\item isAsian - parametr okre�laj�cy typ opcji paryskiej 0 - \emph{Parisian}, 1 - \emph{Parasian} (poprawne warto�ci: 0, 1)
\end{itemize}

Dodatkowo poprzez zmienne globalne przekazywane s� parametry metody PDE
\begin{itemize}
\item Mt - liczba punkt�w siatki w wymiarze czasowym
\item Mx - liczba punkt�w siatki w wymiarze przestrzennym
\item dsigma - przyrost volatility opcji u�ywany do obliczania wsp�czynnika \emph{vega}
\end{itemize}

\subsection{Opis funkcji}
W tym podrozdziale zostan� przedstawione sygnatury funkcji s�u��cych do wyznaczenia cen oraz parametr�w greckich instrument�w opisanych w poprzednim rozdziale. Ka�da z nich zwaraca wektor 7 elemetowy zawieraj�cy kolejno cen�, delt� spot, delt� forward, gamme spot, gamme forward, thet� oraz veg�. 
\subsubsection{DM\_0ut}
Funcja DM\_out s�u�y do wyznaczenia ceny i parametr�w greckich opcji dyskretnie monitorowanych \emph{up and out}, \emph{down and out} 
\begin{align*}
DM\_out(F\_bid, F\_ask, barrier, strike,monitoring\_dates,issue\_date,expire\_date,PPO,OSO \\
,price\_type,barrier\_type, payoff\_type)
\end{align*}
\subsubsection{DM\_in}
Funcja DM\_in s�u�y do wyznaczenia ceny i parametr�w greckich opcji dyskretnie monitorowanych \emph{up and in}, \emph{down and in} 
\begin{align*}
DM\_in(F\_bid, F\_ask, barrier, strike,monitoring\_dates,issue\_date,expire\_date,PPO,OSO,\\price\_type,barrier\_type, payoff\_type)
\end{align*}

\subsubsection{Double\_KO}
Funcja Double\_KO s�u�y do wyznaczenia ceny i parametr�w greckich opcji z podw�jn� barier� typu \emph{out}  monitorowanych dyskretnie lub w spos�b ci�g�y (nale�y wywo�a� funkcj� z parametrem monitoring\_dates = []).  
\begin{align*}
DoubleKO(F\_bid, F\_ask, Lbarrier, Ubarrier, strike,monitoring\_dates,issue\_date,expire\_date,\\ PPO,OSO,price\_type,payoff\_type)
\end{align*}

\subsubsection{KIKO}
Funcja KIKO s�u�y do wyznaczenia ceny i parametr�w greckich opcji z jedn� barier� typu \emph{in}, i z drug� barier� typu \emph{out} monitorowanych dyskretnie lub w spos�b ci�g�y (nale�y wywo�a� funkcj� z parametrem monitoring\_dates = []).  
\begin{align*}
KIKO(F\_bid, F\_ask, Lbarrier, Ubarrier, strike,monitoring\_dates,issue\_date,expire\_date,PPO,OSO\\,price\_type, payoff\_type)
\end{align*}

\subsubsection{Window\_out}
Funcja Window\_out s�u�y do wyznaczenia ceny i parametr�w greckich opcji z jedn� barier� okienkow� typu \emph{out}.  
\begin{align*}
Window\_out(F\_bid, F\_ask, barrier, strike, issue\_date, window\_start\_date, window\_end\_date,\\expire\_date,PPO,OSO,price\_type, barrier\_type, payoff\_type)
\end{align*}

\subsubsection{Window\_in}
Funcja Window\_in s�u�y do wyznaczenia ceny i parametr�w greckich opcji z jedn� barier� okienkow� typu \emph{out}.  
\begin{align*}
Window\_in(F\_bid, F\_ask, barrier, strike, issue\_date, window\_start\_date, window\_end\_date,\\expire\_date,PPO,OSO,price\_type, barrier\_type, payoff\_type)
\end{align*}


\subsubsection{Window\_DoubleKO}
Funcja Window\_DoubleKO s�u�y do wyznaczenia ceny i parametr�w greckich opcji z podw�jn� barier� typu \emph{out} aplikowan� w zdefiniowanym przez parametry oknie czasowym.   
\begin{align*}
Window\_DoubleKO(F\_bid, F\_ask, Lbarrier, Ubarrier, strike,issue\_date, window\_start\_date, \\window\_end\_date, expire\_date,PPO,OSO,price\_type, payoff\_type)
\end{align*}


\subsubsection{Window\_KIKO}
Funcja Window\_KIKO s�u�y do wyznaczenia ceny i parametr�w greckich opcji z jedn� barier� typu \emph{in}, i z drug� barier� typu \emph{out} aplikowanymi w wyznaczonym przez parametry oknie czasowym. 
\begin{align*}
Window\_KIKO(F\_bid, F\_ask, Lbarrier, Ubarrier, strike,issue\_date, window\_start\_date,\\ window\_end\_date, expire\_date,PPO,OSO,price\_type, payoff\_type)
\end{align*}

\subsubsection{CalculatePriceGreeksParisianOut}
Funcja CalculatePriceGreeksParisianOut s�u�y do wyznaczenia ceny i parametr�w greckich opcji typu paryskiego z barier� typu \emph{out}. 
\begin{align*}
CalculatePriceGreeksParisianOut(F\_bid, F\_ask, barrier,day\_hat, strike,issue\_date,expire\_date, \\ PPO,OSO,price\_type, barrier\_type, payoff\_type,isAsian)
\end{align*}

\subsubsection{CalculatePriceGreeksParisianIn}
Funcja CalculatePriceGreeksParisianIn s�u�y do wyznaczenia ceny i parametr�w greckich opcji typu paryskiego z barier� typu \emph{in}. 
\begin{align*}
CalculatePriceGreeksParisianIn(F\_bid, F\_ask, barrier,day\_hat, strike,issue\_date,expire\_date,\\ PPO,OSO,price\_type, barrier\_type, payoff\_type,isAsian)
\end{align*}





\begin{thebibliography}{99}
\addcontentsline{toc}{chapter}{Bibliografia}

\bibitem[Bea65]{beaman} Juliusz Beaman, \textit{Morbidity of the Jolly
    function}, Mathematica Absurdica, 117 (1965) 338--9.

\bibitem[Blar16]{eb1} Elizjusz Blarbarucki, \textit{O pewnych
    aspektach pewnych aspekt�w}, Astrolog Polski, Zeszyt 16, Warszawa
  1916.


\end{thebibliography}

\end{document}


%%% Local Variables:
%%% mode: latex
%%% TeX-master: t
%%% coding: latin-2
%%% End:
