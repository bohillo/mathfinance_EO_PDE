\documentclass{beamer}
\usetheme{Warsaw}
\usepackage{textcomp}
\usepackage{times}
\usepackage{amsmath}
\usepackage{amsfonts}
\usepackage{amssymb}
\usepackage{amsthm}
\usepackage[T1]{fontenc}
\usepackage[utf8]{inputenc}
\usepackage{graphicx}
\usepackage{xcolor}
%\usepackage{subfigure}
%\usepackage{enumitem}
\usepackage[polish]{babel}
\usepackage{geometry}

\setbeamertemplate{itemize item}{\scriptsize\raise1.25pt\hbox{\donotcoloroutermaths$\blacktriangleright$}}
\setbeamertemplate{itemize subitem}{\tiny\raise1.5pt\hbox{\donotcoloroutermaths$\blacktriangleright$}}
\setbeamertemplate{itemize subsubitem}{\tiny\raise1.5pt\hbox{\donotcoloroutermaths$\blacktriangleright$}}
\setbeamertemplate{enumerate item}{\insertenumlabel.}
\setbeamertemplate{enumerate subitem}{\insertenumlabel.\insertsubenumlabel}
\setbeamertemplate{enumerate subsubitem}{\insertenumlabel.\insertsubenumlabel.\insertsubsubenumlabel}
\setbeamertemplate{enumerate mini template}{\insertenumlabel}
\usepackage{bbm}
\usepackage{amsmath}


\usepackage{listings}
\lstset{language=R,%
   numbers=left,%
   tabsize=3,%
   numberstyle=\footnotesize,%
   basicstyle=\ttfamily \footnotesize \color{black},%
   escapeinside={(*@}{@*)}}
	
\usepackage{hyperref}
\hypersetup{colorlinks=true,
            linkcolor=black,
            citecolor=darkgray,
            urlcolor=darkgray,
            pagecolor=darkgray}

\usepackage{longtable}
% tweaks by Michał Ramsza (changing citations)
\makeatletter
\renewcommand\@biblabel[1]{}
\def\@cite#1#2{{#1\if@tempswa:#2\fi}}
%\def\@cite#1#2{\unskip\textsuperscript{#1\if@tempswa, #2\fi}} 
\makeatother


% --- wlasne komendy --------------------------------------------------
\DeclareMathOperator*{\argmax}{\mathrm{argmax}}
\DeclareMathOperator*{\argmin}{\mathrm{argmin}}
\DeclareMathOperator*{\qa}{\forall}
\DeclareMathOperator*{\qe}{\exists}
\newcommand{\setR}{\mathbb{R}}
\newcommand{\setN}{\mathbb{N}}
\newcommand{\trans}[1]{{#1}^{\mathrm{T}}}
\newcommand{\e}{\mathrm{e}}
\newcommand{\sgn}[1]{\mathrm{sgn}\left( #1 \right)}
\newcommand{\T}[1]{T\left(#1\right)}
\newcommand{\Det}[1]{\mathrm{det}\left( #1\right)}
\newcommand{\Tr}[1]{\mathrm{tr}\left( #1\right)}
\newcommand{\I}{\mathrm{I}}

\author{Piotr Bochnia, Paweł Marcinkowski}
\title{Finanse obliczeniowe - Duży projekt \\ Wycena skomplikowanych opcji barierowych metodą PDE}

\begin{document}
\begin{frame}
  \titlepage
\end{frame}
\begin{frame}
  \frametitle{Wyceniane instrumenty}

  \begin{itemize}
  \item Opcje z barierą monitorowaną dyskretnie
  \item Opcje z barierą monitorowaną w oknie czasowym
  \item Opcje paryskie
  \end{itemize}
\end{frame}

\begin{frame}
\frametitle{Opcje z barierą monitorowaną dyskretnie}
Niech $0 \leq T_1 < T_2 < \ldots < T_L \leq T$ będą punktami monitorowania bariery. 
\begin{itemize}

\item \emph{Up and out} z wypłatą $X = f(S_T) \cdot \mathbbm{1}_{\{\forall_{ t \in \{T_1, \ldots, T_L\} } S_t < U \}}$
\item \emph{Up and in} z wypłatą $X = f(S_T) \cdot \mathbbm{1}_{\{\exists_{ t \in \{T_1, \ldots, T_L\} } S_t \geq U \}}$
\item \emph{Down and out} z wypłatą $X = f(S_T) \cdot \mathbbm{1}_{\{\forall_{ t \in \{T_1, \ldots, T_L\} } S_t > L \}}$
\item \emph{Down and in} z wypłatą $X = f(S_T) \cdot \mathbbm{1}_{\{\exists_{ t \in \{T_1, \ldots, T_L\} } S_t \leq L \}}$
\item \emph{Double Knock-out} z wypłatą $X = f(S_T) \cdot \mathbbm{1}_{\{\forall_{ t \in \{T_1, \ldots, T_L\} } L < S_t  < U \}}$.
\item \emph{Knock-in Knock-out} z wypłatą $X = f(S_T) \cdot \left( \mathbbm{1}_{\{\exists_{ t \in \{T_1, \ldots, T_L\} } S_t \leq L \} \wedge \forall_{ t \in \{T_1, \ldots, T_L\} } S_t < U \}}\right)$.  

\end{itemize}


\end{frame}

\begin{frame}
\frametitle{Opcje z barierą monitorowaną w oknie czasowym}
  Niech $0 \leq \tau_1 < \tau_2 \leq T$ będą punktami odpowiednio początku i końca okna, którym monitorowana jest bariera. 
\begin{itemize}
\item \emph{Up and out} z wypłatą $X = f(S_T) \cdot \mathbbm{1}_{\{\forall_{ t \in \left[ \tau_1, \tau_2 \right] } S_t < U \}}$
\item \emph{Up and in} z wypłatą $X = f(S_T) \cdot \mathbbm{1}_{\{\exists_{ t \in \left[ \tau_1, \tau_2 \right] } S_t \geq U \}}$
\item \emph{Down and out} z wypłatą $X = f(S_T) \cdot \mathbbm{1}_{\{\forall_{ t \in \left[ \tau_1, \tau_2 \right] } S_t > L \}}$
\item \emph{Down and in} z wypłatą $X = f(S_T) \cdot \mathbbm{1}_{\{\exists_{ t \in \left[ \tau_1, \tau_2 \right] } S_t \leq L \}}$
\item \emph{Double Knock-out} z wypłatą $X = f(S_T) \cdot \mathbbm{1}_{\{\forall_{ t \in \left[ \tau_1, \tau_2 \right] } L < S_t  < U \}}$
\item \emph{Knock-in Knock-out} z wypłatą $X = f(S_T) \cdot \left( \mathbbm{1}_{\{\exists_{ t \in \left[ \tau_1, \tau_2 \right] } S_t \leq L \} \wedge \forall_{ t \in \left[ \tau_1, \tau_2 \right] } S_t < U \}}\right)$

\end{itemize}
\end{frame}

\begin{frame}
\frametitle{Opcje paryskie}
Jednobarierowe opcje paryskie są kontraktami, w których własność \emph{in} lub \emph{out} jest aktywowana nie w momencie dotknięcia bariery lecz po pewnym ustalonym z góry czasie przebywania ceny instrumentu bazowego nad lub pod barierą (czas barierowy). Opcje typu paryskiego dzielą się na dwie klasy:
\begin{itemize}
\item  \emph{Parisian} - W momencie, gdy cena akcji jest równa barierze, czas barierowy jest zerowany.
\item  \emph{Parasian} - Czas barierowy jest sumą wszyskich przebywań poza barierą.
\end{itemize}
Wypłata z opcji jest równa wypłacie z opcji europejskiej po spełnieniu warunków zależnych od typu bariery (analogicznie do tradycyjnych opcji barierowych) i zero w przeciwnym przypadku.
  
\end{frame}

\begin{frame}
  \frametitle{Funkcje do wyceny}
\begin{itemize}
  
\item{DM\_0ut}

\item{DM\_in}

\item{Double\_KO}

\item{KIKO}

\item{Window\_out}

\item{Window\_in}

\item{Window\_DoubleKO}

\item{Window\_KIKO}
\item{CalculatePriceGreeksParisianOut}

\item{CalculatePriceGreeksParisianIn}
  \end{itemize}
\end{frame}

\begin{frame}
  \frametitle{Input - Argumenty funkcji (1)}
\begin{itemize}
\item F\_bid - kurs forward bid
\item F\_bid - kurs forward ask
\item barrier - wysokość bariery (w przypadku opcji jednobarierowych, wyrażona w walucie kwotowania)
\item Lbarrier - wysokość dolnej bariery (w przypadku opcji dwubarierowych, wyrażona w walucie kwotowania)
\item Ubarrier - wysokość górnej bariery (w przypadku opcji dwubarierowych, wyrażona w walucie kwotowania)
\item strike - kurs wykonania opcji
\item barrier\_type - typ bariery (w przypadku opcji jednobarierowych, poprawne wartości: \emph{up}, \emph{down})
\item payoff\_type - typ opcji (poprawne wartości: \emph{put}, \emph{call})
\end{itemize}

\end{frame}

\begin{frame}
  \frametitle{Input - Argumenty funkcji (2)}
\begin{itemize}

\item issue\_date - data zawarcia kontraktu (np. '21-May-2014')
\item expire\_date - data zapadalności kontraktu (np. '21-May-2014') 
\item PPO - liczba dni roboczych od daty zawarcia kontraktu do dnia zapłaty premi opcyjnej (\emph{Premium Payment Offset})
\item OSO - liczba dni roboczych od daty zapadalności opcji do dnia rozliczenia kontraktu (\emph{Option Settlement Offset})
\item price\_type - typ obliczanej ceny (poprawne wartości: \emph{bid}, \emph{ask})
\end{itemize}

\end{frame}

\begin{frame}
  \frametitle{Input - Parametry przekazywane przez zmienne globalne}

\begin{itemize}
\item Mt - liczba punktów siatki w wymiarze czasowym
\item Mx - liczba punktów siatki w wymiarze przestrzennym
\item dsigma - przyrost volatility opcji używany do obliczania współczynnika \emph{vega}
\end{itemize}

\end{frame}

\begin{frame}
  \frametitle{Output}

7-elementowy wektor zawierający na kolejnych pozycjach odpowiednio:

\begin{enumerate}
\item  cenę opcji \item deltę spot\item deltę forward\item gammę spot\item gammę forward\item thetę \item vegę
\end{enumerate}
\end{frame}


\begin{frame}
  \frametitle{Funkcje do wyceny - Opcje z barierą monitorowaną dyskretnie}

  \begin{itemize}
  \item \small \textsc{DM\_out(F\_bid, F\_ask, barrier, strike, monitoring\_dates, issue\_date, expire\_date, PPO, OSO, price\_type, barrier\_type, payoff\_type)}
  \item \small \textsc{DM\_in(F\_bid, F\_ask, barrier, strike, monitoring\_dates,issue\_date,expire\_date, PPO, OSO, price\_type,barrier\_type, payoff\_type)}
  \item \small \textsc{DoubleKO(F\_bid, F\_ask, Lbarrier, Ubarrier, strike,monitoring\_dates, issue\_date, expire\_date, PPO, OSO, price\_type,payoff\_type)}
  \item \small \textsc{KIKO(F\_bid, F\_ask, Lbarrier, Ubarrier, strike,monitoring\_dates,issue\_date, expire\_date,PPO,OSO, price\_type, payoff\_type)}
  \end{itemize}

\end{frame}



\begin{frame}
  \frametitle{Funkcje do wyceny - Opcje z barierą monitorowaną dyskretnie - Dodatkowe argumenty}

\begin{itemize}
\item monitoring\_dates - wektor zawierający daty monitorowania barier w kolejności chronologicznej
\end{itemize}

\end{frame}



\begin{frame}
  \frametitle{Funkcje do wyceny - Opcje z barierą monitorowaną w oknie czasowym}
  \begin{itemize}
  \item \small \textsc{Window\_out(F\_bid, F\_ask, barrier, strike, issue\_date, window\_start\_date, window\_end\_date,expire\_date,PPO,OSO,price\_type, barrier\_type, payoff\_type)}
  \item \small \textsc{Window\_in(F\_bid, F\_ask, barrier, strike, issue\_date, window\_start\_date, window\_end\_date,expire\_date,PPO,OSO,price\_type, barrier\_type, payoff\_type)
}
  \item \small \textsc{Window\_DoubleKO(F\_bid, F\_ask, Lbarrier, Ubarrier, strike,issue\_date, window\_start\_date, window\_end\_date, expire\_date,PPO,OSO,price\_type, payoff\_type)}
  \item \small \textsc{Window\_KIKO(F\_bid, F\_ask, Lbarrier, Ubarrier, strike,issue\_date, window\_start\_date, window\_end\_date, expire\_date,PPO,OSO,price\_type, payoff\_type}

  \end{itemize}
\end{frame}

\begin{frame}
  \frametitle{Funkcje do wyceny - Opcje z barierą monitorowaną w oknie czasowym - Dodatkowe argumenty}

\begin{itemize}
\item window\_start\_date - data początku okna czasowego, w którym monitorowane są bariery
\item window\_end\_date - data końca okna czasowego, w którym monitorowane są bariery
\end{itemize}

\end{frame}



\begin{frame}
  \frametitle{Funkcje do wyceny - Opcje paryskie}
  \begin{itemize}
  \item \small \textsc{CalculatePriceGreeksParisianOut(F\_bid, F\_ask, barrier,day\_hat, strike,issue\_date,expire\_date,PPO, OSO, price\_type, barrier\_type, payoff\_type,isAsian)}
  \item \small \textsc{CalculatePriceGreeksParisianIn(F\_bid, F\_ask, barrier,day\_hat, strike,issue\_date,expire\_date, PPO, OSO, price\_type, barrier\_type, payoff\_type,isAsian)}

  \end{itemize}
\end{frame}

\begin{frame}
  \frametitle{Funkcje do wyceny - Opcje paryskie - Dodatkowe argumenty}

\begin{itemize}
\item day\_hat - czas przebywania poza barierą konieczny do aktywacji bariery (w przypadku opcji paryskich, wyrażony w dniach)
\item isAsian - parametr określający typ opcji paryskiej 0 - \emph{Parisian}, 1 - \emph{Parasian} (poprawne wartości: 0, 1)
\end{itemize}

\end{frame}
\end{document}