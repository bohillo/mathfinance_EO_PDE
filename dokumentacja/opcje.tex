\section{Wyceniane instrumenty} Wycen� przeprowadzono dla wybranych
skomplikowanych opcji barierowych: opcji z pojedyncz�, dyskretnie
monitorowan� barier�, opcji z podw�jn� barier� (monitorowan� w spos�b
zar�wno dyskretny jak i ci�g�y), opcji barierowych z barierami
monitorowanymi w oknie czasowym, oraz opcji paryskich.

Poni�ej wymieniono poszczeg�lne typy kontrakt�w, dla kt�rych zaimplementowano algorytm wyceny. Niech $T$ b�dzie czasem zapadalno�ci opcji, $K$ cen� wykonania, $U$ i $L$ odpowiednio barier� g�rn� i doln�, oraz $S_t$ cen� instrumentu bazowego w chwili $t$. Ponadto niech $f$ b�dzie funkcj� wyp�aty dla opcji waniliowej tj. $f(x) = (x - K) ^{+}

\subsection{Opcje z barier� monitorowan� dyskretnie}
Niech $0 \leq T_1 < T_2 < \ldots < T_L \leq T$ b�d� punktami monitorowania bariery. 
\begin{itemize}
\item \emph{Up and out} z wyp�at� $X = f(S_T) \cdot \mathbbm{1}_{\{\forall_{ t \in \{T_1, \ldots, T_L\} } S_t < U \}}$
\item \emph{Up and in} z wyp�at� $X = f(S_T) \cdot \mathbbm{1}_{\{\exists_{ t \in \{T_1, \ldots, T_L\} } S_t \geq U \}}$
\item \emph{Down and out} z wyp�at� $X = f(S_T) \cdot \mathbbm{1}_{\{\forall_{ t \in \{T_1, \ldots, T_L\} } S_t > L \}}$
\item \emph{Down and in} z wyp�at� $X = f(S_T) \cdot \mathbbm{1}_{\{\exists_{ t \in \{T_1, \ldots, T_L\} } S_t \leq L \}}$
\item \emph{Double Knock-out} z wyp�at� $X = f(S_T) \cdot \mathbbm{1}_{\{\forall_{ t \in \{T_1, \ldots, T_L\} } L < S_t  < U \}}$
\item \emph{Knock-in Knock-out} z wyp�at� $X = f(S_T) \cdot \left( \mathbbm{1}_{\{\exists_{ t \in \{T_1, \ldots, T_L\} } S_t \leq L \} \wedge \forall_{ t \in \{T_1, \ldots, T_L\} } S_t < U \}}\right)$


\end{itemize}

\subsection{Opcje z barier� monitorowan� w oknie czasowym}

\subsection{Opcje paryskie}





