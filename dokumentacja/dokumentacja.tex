\documentclass{article}

\usepackage{polski}
\usepackage{xcolor}
\usepackage{bbm}

\usepackage[latin2]{inputenc}

% Dane magistranta:

\author{Piotr Bochnia, Pawe� Marcinkowski}


\title{Finanse obliczeniowe - Du�y projekt \\ Wycena skomplikowanych opcji barierowych metod� PDE}



% Tu jest dobre miejsce na Twoje w�asne makra i~�rodowiska:
\newtheorem{defi}{Definicja}[section]
\newcommand{\notka}[1]{\textcolor{red}{#1}}
\newtheorem{assum}{Za�o�enie}[section]
% koniec definicji

\begin{document}
\maketitle

\clearpage
\tableofcontents
%\listoffigures
%\listoftables
\clearpage


\section{Wst�p}
Celem niniejszego projektu jest implementacja algorytm�w wyceny wybranych opcji barierowych metod� opart� na rozwi�zywaniu r�wnania Blacka-Scholesa. Na podstawie danych rynkowych oraz charakterystyk opcji napisany w Octave program wyznacza parametry r�wnania Blacka-Scholesa wraz z odpowiednimi dla danego kontraktu warunkami brzegowymi i ko�cowymi, a nast�pnie rozwi�zuje to r�wnanie metod� r�nic sko�czonych (schemat Crank-Nicholson). Poza cen� opcji obliczane s� tak�e parametry greckie: \emph{delta spot}, \emph{delta forward}, \emph{gamma spot}, \emph{gamma forward}, \emph{theta}, oraz \emph{vega}.  
\section{Wyceniane instrumenty} Wycen� przeprowadzono dla wybranych
skomplikowanych opcji barierowych: opcji z pojedyncz�, dyskretnie
monitorowan� barier�, opcji z podw�jn� barier� (monitorowan� w spos�b
zar�wno dyskretny jak i ci�g�y), opcji barierowych z barierami
monitorowanymi w oknie czasowym, oraz opcji paryskich.

Poni�ej wymieniono poszczeg�lne typy kontrakt�w, dla kt�rych zaimplementowano algorytm wyceny. Niech $T$ b�dzie czasem zapadalno�ci opcji, $K$ cen� wykonania, $U$ i $L$ odpowiednio barier� g�rn� i doln�, oraz $S_t$ cen� instrumentu bazowego w chwili $t$. Ponadto niech $f$ b�dzie funkcj� wyp�aty dla opcji waniliowej tj. $f(x) = (x - K) ^{+}$ dla opcji \emph{call} i $f(x) = (K - x) ^{+}$ dla opcji \emph{put}.

\subsection{Opcje z barier� monitorowan� dyskretnie}
Niech $0 \leq T_1 < T_2 < \ldots < T_L \leq T$ b�d� punktami monitorowania bariery. 
\begin{itemize}
\item \emph{Up and out} z wyp�at� $X = f(S_T) \cdot \mathbbm{1}_{\{\forall_{ t \in \{T_1, \ldots, T_L\} } S_t < U \}}$
\item \emph{Up and in} z wyp�at� $X = f(S_T) \cdot \mathbbm{1}_{\{\exists_{ t \in \{T_1, \ldots, T_L\} } S_t \geq U \}}$
\item \emph{Down and out} z wyp�at� $X = f(S_T) \cdot \mathbbm{1}_{\{\forall_{ t \in \{T_1, \ldots, T_L\} } S_t > L \}}$
\item \emph{Down and in} z wyp�at� $X = f(S_T) \cdot \mathbbm{1}_{\{\exists_{ t \in \{T_1, \ldots, T_L\} } S_t \leq L \}}$
\item \emph{Double Knock-out} z wyp�at� $X = f(S_T) \cdot \mathbbm{1}_{\{\forall_{ t \in \{T_1, \ldots, T_L\} } L < S_t  < U \}}$
\item \emph{Knock-in Knock-out} z wyp�at� $X = f(S_T) \cdot \left( \mathbbm{1}_{\{\exists_{ t \in \{T_1, \ldots, T_L\} } S_t \leq L \} \wedge \forall_{ t \in \{T_1, \ldots, T_L\} } S_t < U \}}\right)$


\end{itemize}

\subsection{Opcje z barier� monitorowan� w oknie czasowym}
Niech $0 \leq \tau_1 < \tau_2 \leq T$ b�d� punktami odpowiednio pocz�tku i ko�ca okna, kt�rym monitorowana jest bariera. 
\begin{itemize}
\item \emph{Up and out} z wyp�at� $X = f(S_T) \cdot \mathbbm{1}_{\{\forall_{ t \in \left[ \tau_1, \tau_2 \right] } S_t < U \}}$
\item \emph{Up and in} z wyp�at� $X = f(S_T) \cdot \mathbbm{1}_{\{\exists_{ t \in \left[ \tau_1, \tau_2 \right] } S_t \geq U \}}$
\item \emph{Down and out} z wyp�at� $X = f(S_T) \cdot \mathbbm{1}_{\{\forall_{ t \in \left[ \tau_1, \tau_2 \right] } S_t > L \}}$
\item \emph{Down and in} z wyp�at� $X = f(S_T) \cdot \mathbbm{1}_{\{\exists_{ t \in \left[ \tau_1, \tau_2 \right] } S_t \leq L \}}$
\item \emph{Double Knock-out} z wyp�at� $X = f(S_T) \cdot \mathbbm{1}_{\{\forall_{ t \in \left[ \tau_1, \tau_2 \right] } L < S_t  < U \}}$
\item \emph{Knock-in Knock-out} z wyp�at� $X = f(S_T) \cdot \left( \mathbbm{1}_{\{\exists_{ t \in \left[ \tau_1, \tau_2 \right] } S_t \leq L \} \wedge \forall_{ t \in \left[ \tau_1, \tau_2 \right] } S_t < U \}}\right)$


\end{itemize}
\subsection{Opcje paryskie}
Jednobarierowe opcje paryskie s� kontraktami, w kt�rych w�asno�� \emph{in} lub \emph{out} jest aktywowana nie w momencie dotkni�cia bariery lecz po pewnym ustalonym z g�ry czasie przebywania ceny instrumentu bazowego nad lub pod barier�. Opcje typu paryskiego dziel� si� na dwie klasy:
\begin{itemize}
\item  \emph{Parisian} - 
\end{itemize}

\section{Dokumentacja funkcji}
\subsection{Opis parametrów}
\subsection{Opis funkcji}
W tym podrozdziale zostaną przedstawione sygnatury funkcji służących do wyznaczenia cen oraz parametrów greckich instrumentów opisanych w poprzednim rozdziale. Każda z nich zwaraca wektor 7 elemetowy zawierający kolejno cenę, deltę spot, deltę forward, gamme spot, gamme forward, thetę oraz vegę. 
\subsubsection{DM\_0ut}
Funcja DM\_out służy do wyznaczenia ceny i parametrów greckich opcji dyskretnie monitorowanych \emph{up and out}, \emph{down and out} 
\begin{displaymath}
DM\_out(F\_bid, F\_ask, barrier, strike,monitoring\_dates,issue\_date,expire\_date,PPO,OSO,price\_type,barrier\_type, payoff\_type)
\end{displaymath}
\subsubsection{DM\_in}
Funcja DM\_in służy do wyznaczenia ceny i parametrów greckich opcji dyskretnie monitorowanych \emph{up and in}, \emph{down and in} 
\begin{displaymath}
DM\_in(F\_bid, F\_ask, barrier, strike,monitoring\_dates,issue\_date,expire\_date,PPO,OSO,price\_type,barrier\_type, payoff\_type)
\end{displaymath}

\subsubsection{Double\_KO}
Funcja Double\_KO służy do wyznaczenia ceny i parametrów greckich opcji z podwójną barierą typu \emph{out}  monitorowanych dyskretnie lub w sposób ciągły (należy wywołać funkcję z parametrem monitoring\_dates = []).  
\begin{displaymath}
DoubleKO(F\_bid, F\_ask, Lbarrier, Ubarrier, strike,monitoring\_dates,issue\_date,expire\_date,PPO,OSO,price\_type,payoff\_type)
\end{displaymath}

\subsubsection{KIKO}
Funcja KIKO służy do wyznaczenia ceny i parametrów greckich opcji z jedną barierą typu \emph{in}, i z drugą barierą typu \emph{out} monitorowanych dyskretnie lub w sposób ciągły (należy wywołać funkcję z parametrem monitoring\_dates = []).  
\begin{displaymath}
KIKO(F\_bid, F\_ask, Lbarrier, Ubarrier, strike,monitoring\_dates,issue\_date,expire\_date,PPO,OSO,price\_type, payoff\_type)
\end{displaymath}

\subsubsection{Window\_out}
Funcja Window\_out służy do wyznaczenia ceny i parametrów greckich opcji z jedną barierą okienkową typu \emph{out}.  
\begin{displaymath}
Window\_out(F\_bid, F\_ask, barrier, strike, issue\_date, window\_start\_date, window\_end\_date,expire\_date,PPO,OSO,price\_type, barrier\_type, payoff\_type)
\end{displaymath}

\subsubsection{Window\_in}
Funcja Window\_in służy do wyznaczenia ceny i parametrów greckich opcji z jedną barierą okienkową typu \emph{out}.  
\begin{displaymath}
Window\_in(F\_bid, F\_ask, barrier, strike, issue\_date, window\_start\_date, window\_end\_date,expire\_date,PPO,OSO,price\_type, barrier\_type, payoff\_type)
\end{displaymath}


\subsubsection{Window\_DoubleKO}
Funcja Window\_DoubleKO służy do wyznaczenia ceny i parametrów greckich opcji z podwójną barierą typu \emph{out} aplikowaną w zdefiniowanym przez parametry oknie czasowym.   
\begin{displaymath}
Window\_DoubleKO(F\_bid, F\_ask, Lbarrier, Ubarrier, strike,issue\_date, window\_start\_date, window\_end\_date, expire\_date,PPO,OSO,price\_type, payoff\_type)
\end{displaymath}


\subsubsection{Window\_KIKO}
Funcja Window\_KIKO służy do wyznaczenia ceny i parametrów greckich opcji z jedną barierą typu \emph{in}, i z drugą barierą typu \emph{out} aplikowanymi w wyznaczonym przez parametry oknie czasowym. 
\begin{displaymath}
Window\_KIKO(F\_bid, F\_ask, Lbarrier, Ubarrier, strike,issue\_date, window\_start\_date, window\_end\_date, expire\_date,PPO,OSO,price\_type, payoff\_type)
\end{displaymath}

\subsubsection{CalculatePriceGreeksParisianOut}
Funcja CalculatePriceGreeksParisianOut służy do wyznaczenia ceny i parametrów greckich opcji typu paryskiego z barierą typu \emph{out}. 
\begin{displaymath}
CalculatePriceGreeksParisianOut(F\_bid, F\_ask, barrier,day\_hat, strike,issue\_date,expire\_date,PPO,OSO,price\_type, barrier\_type, payoff\_type,isAsian)
\end{displaymath}

\subsubsection{CalculatePriceGreeksParisianIn}
Funcja CalculatePriceGreeksParisianIn służy do wyznaczenia ceny i parametrów greckich opcji typu paryskiego z barierą typu \emph{in}. 
\begin{displaymath}
CalculatePriceGreeksParisianIn(F\_bid, F\_ask, barrier,day\_hat, strike,issue\_date,expire\_date,PPO,OSO,price\_type, barrier\_type, payoff\_type,isAsian)
\end{displaymath}






\begin{thebibliography}{99}
\addcontentsline{toc}{chapter}{Bibliografia}

\bibitem[Bea65]{beaman} Juliusz Beaman, \textit{Morbidity of the Jolly
    function}, Mathematica Absurdica, 117 (1965) 338--9.

\bibitem[Blar16]{eb1} Elizjusz Blarbarucki, \textit{O pewnych
    aspektach pewnych aspekt�w}, Astrolog Polski, Zeszyt 16, Warszawa
  1916.


\end{thebibliography}

\end{document}


%%% Local Variables:
%%% mode: latex
%%% TeX-master: t
%%% coding: latin-2
%%% End:
