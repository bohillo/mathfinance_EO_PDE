\section{Dokumentacja funkcji}
Wszystkie funkcje u�ywane do wyceny opcji zwracaj� 7-elementowy wektor zawieraj�cy na kolejnych pozycjach odpowiednio cen� opcji, delt� spot, delt� forward, gamm� spot, gamm� forward, thet� oraz veg�. Funkcje przyjmuj� nast�puj�ce argumenty:

\begin{itemize}
\item F\_bid - kurs forward bid
\item F\_bid - kurs forward ask
\item barrier - wysoko�� bariery (w przypadku opcji jednobarierowych, wyra�ona w walucie kwotowania)
\item Lbarrier - wysoko�� dolnej bariery (w przypadku opcji dwubarierowych, wyra�ona w walucie kwotowania)
\item Ubarrier - wysoko�� g�rnej bariery (w przypadku opcji dwubarierowych, wyra�ona w walucie kwotowania)
\item strike - kurs wykonania opcji
\item issue\_date - data zawarcia kontraktu (np. '21-May-2014')
\item expire\_date - data zapadalno�ci kontraktu (np. '21-May-2014') 
\item PPO - liczba dni roboczych od daty zawarcia kontraktu do dnia zap�aty premi opcyjnej (\emph{Premium Payment Offset})
\item OSO - liczba dni roboczych od daty zapadalno�ci opcji do dnia rozliczenia kontraktu (\emph{Option Settlement Offset})
\item monitoring\_dates - wektor zawieraj�cy daty monitorowania barier w kolejno�ci chronologicznej
\item price\_type - typ obliczanej ceny (poprawne warto�ci: \emph{bid}, \emph{ask})
\item barrier\_type - typ bariery (w przypadku opcji jednobarierowych, poprawne warto�ci: \emph{up}, \emph{down})
\item payoff\_type - typ opcji (poprawne warto�ci: \emph{put}, \emph{call})
\item window\_start\_date - data pocz�tku okna czasowego, w kt�rym monitorowane s� bariery
\item window\_end\_date - data ko�ca okna czasowego, w kt�rym monitorowane s� bariery
\item day\_hat - czas przebywania poza barier� konieczny do aktywacji bariery (w przypadku opcji paryskich, wyra�ony w dniach)
\item isAsian - parametr okre�laj�cy typ opcji paryskiej 0 - \emph{Parisian}, 1 - \emph{Parasian} (poprawne warto�ci: 0, 1)
\end{itemize}

Dodatkowo poprzez zmienne globalne przekazywane s� parametry metody PDE
\begin{itemize}
\item Mt - liczba punkt�w siatki w wymiarze czasowym
\item Mx - liczba punkt�w siatki w wymiarze przestrzennym
\item dsigma - przyrost volatility opcji u�ywany do obliczania wsp�czynnika \emph{vega}
\end{itemize}